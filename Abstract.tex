\documentclass[12pt,a4paper]{article}
\usepackage[utf8x]{luainputenc}
\usepackage{ucs}
\usepackage[german]{babel}
\usepackage[left=2cm,right=2cm,top=2cm,bottom=2cm]{geometry}
\usepackage{url}
\usepackage{hyperref}
\author{Robert Hemstedt\\ \small{Entwickler bei Miaplaza Inc.} }
\title{Entwicklung von Educational Mobile Games -- Einblick in das Start-Up Miaplaza Inc.}
\date{}
\begin{document}
\maketitle
\begin{abstract}
In einem Panel würde ich anhand von Apps des amerikanischen Unternehmens \linebreak Miaplaza Inc., aufzeigen, welche Schwierigkeiten es gab, neu ins App-Geschäft einzusteigen und welche positiven und negativen Erfahrungen gemacht wurden sowie bei der Entwicklung von Apps darlegen, wie neben Grafik-, Programmier- und Game\-design\-kennt\-nis\-sen auch moderne Methoden des Lernens zur Optimierung des Lernprozesses einfließen. Anschließend soll eine Diskussion über e-Learning durch Educational Games folgen.
\end{abstract}
Miaplaza Inc. bietet Online-Lernplattformen für Kinder an und ist 2013 neu ins App-Geschäft eingestiegen und bietet seither verschiedene Formate von Educational Games im Apple App Store an.

Zuerst wurde eine Quiz-App für Kinder mehrfach outgesourced, anschließend innerhalb des eigenen Unternehmen weiterentwickelt. Diese bot anfangs für die Lernende ein Reward-Game an, das aber kaum genutzt wurde, obwohl das Reward-Learning-Prinzip eines der Value Propositions von Miaplaza's Educational Websites\footnote{\url{www.always-icecream.com} und \url{www.clever-dragons.com} } ist. 

In die Quiz-App floss dann Wissen um gezieltes Memoisieren ein, um den Lernprozess des Users zu verbessern. Eine der erfolgreichen Apps dieses Formats ist \glqq Vocabulary Practice: Greek and Latin Root Words Vocabulary Game\grqq\footnote{\url{http://bit.ly/1jmvxJg}}. 

Um Fuß im App Store zu fassen, wurden viele solcher Quiz-Apps veröffentlich, die sich zumeist nur durch die vermittelten Inhalte unterscheiden. Damit sollte eine gute Findbarkeit im App Store erzielt werden. Hiermit begann dann die Auseinandersetzung mit dem Problem der App Store Optimization (ASO).
\newline

Vorstellen möchte ich auch die maßgeblich von mir entwickelte App \glqq GeoTouch\grqq\footnote{\label{geotouch}\url{http://bit.ly/1jBQ3WD}}, zum Lernen der Lage und Formen von Staaten der USA und Ländern der Welt. Hier flossen bereits bei der Spiel- und  Designkonzeption eine Analyse von Konkurrenzprodukten, Erfahrungen aus einem ähnlichen Spiel für die Websites\footnote{\url{http://always-icecream.com/publicGeographyGame}} sowie die Diskussionen mit anderen Kollegen ein. 

Durch die Beauftragung eines Grafikers wurde die App optisch ansprechender. Ebenso werden Daten wie Lernraten aus dem Pendant auf den Websites genutzt, um den Lernprozess zu verbessern.

Diese App allein wurde erfolgreicher als der vorherige App-Typus. 
Eine Review, die das Spiel- und Lernprinzip gut beschreibt, lautet \begin{quote}
 \glqq I've been trying to learn my geography [...] and this app is the absolute best way to do it. It shows you a few countries or states then they test you. It keeps on doing that while retesting what you've learn. It's the absolute best for making sure you retain the information. [...] \grqq[\ref{geotouch}], \end{quote}
 
Mittels Feedback von \url{Usertesting.com} und iTunes-Reviews werden auch Verbesserungsvorschläge der Nutzer verwirklicht.

\section*{Über mich}
Ich bin 20 Jahre alt, studiere im vierten Semester Mathematik, B.Sc., an der Rheinischen Friedrich-Wilhelms-Universität Bonn mit Nebenfach Informatik. 

Für Miaplaza Inc. entwickle ich seit Januar 2013 und ich programmiere bereits seit meinem 14. Lebensjahr. Die vorwiegend studentischen Entwickler von Miaplaza Inc. genießen große Freiheit in Entwurf und Umsetzung ihrer Ideen und haben so maßgeblichen Einfluss auf die Gestaltung der Websites und mobilen Apps.
\end{document}